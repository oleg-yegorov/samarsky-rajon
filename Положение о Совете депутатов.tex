\documentclass[a4paper, 11pt]{article}

\usepackage[T2A]{fontenc}
\usepackage[utf8]{inputenc}
\usepackage[russian]{babel}
\usepackage{hyphenat}
\usepackage{indentfirst}

\usepackage{verbatim}
\usepackage{titlesec} % \usepackage[explicit]{titlesec}


\usepackage{hyperref}
\hypersetup{
    colorlinks,
    citecolor=black,
    filecolor=black,
    linkcolor=black,
    urlcolor=black
}

%sentence (add ; and . in enumerate)
%https://tex.stackexchange.com/questions/450715/vertical-list-customization-like-enumitems-itemjoin
\usepackage{enumitem}

\SetEnumitemKey{sentence}{%
  before*=\sentencelistprep,
  after*={\unskip.},
  itemjoin={;},
  label=\asbuk*),
  ref=\asbuk*,
  leftmargin=7ex,
  itemsep=-1ex,
  topsep=0ex, 
  itemjoin={{; }}
}

\SetEnumitemKey{sentenced}{%
  before*=\sentencelistprep,
  after*={\unskip.},
  itemjoin={;},
  ref=\asbuk*,
  leftmargin=7ex,
  itemsep=-1ex,
  topsep=0ex, 
  itemjoin={{; }}
}


%sentence (add ; and . in enumerate)
\AddEnumerateCounter{\asbuk}{\russian@alph}{щ}
\let\sentenceitemjoin\empty
\edef\sentenceitem{\noexpand\sentenceitemjoin\unexpanded\expandafter{\item}}%
\makeatletter %% <- make @ usable in command sequences
\newcommand*\sentencelistprep{%
  \def\sentenceitemjoin{\def\sentenceitemjoin{\unskip\enit@itemjoin}}%
  \let\item\sentenceitem
}
\makeatother  %% <- revert @

\newcommand{\mypart}[2]{
    \titleformat{\part}[display]
    {\bfseries\centering} % before-code
    {} % header
    {10pt} % vertical or horizontal space
    {} % title
    \titlespacing*{\part}{0pt}{0pt}{10pt}
    {\pagebreak\part[#1]{#2}}
}

\newcommand{\mysection}[1]{
    \titleformat{\section}[display]
    {\bfseries\centering} % before-code
    {} % header
    {0pt} % vertical or horizontal space
    {Статья \MakeUppercase{\thesection}.\ } % title
    \titlespacing*{\section}{0pt}{0pt}{-11pt}
    {
        \section{#1}{}
    }
}

\newcommand{\mysectiom}[1]{
    \titleformat{\section}[display]
    {\bfseries\centering} % before-code
    {} % header
    {0pt} % vertical or horizontal space
    {} % title
    \titlespacing*{\section}{0pt}{10pt}{-11pt}
    {
        \phantomsection
        \section*{#1}{}
        \addcontentsline{toc}{section}{\protect{#1}}%
    }
}

\newcommand{\mysubsection}[1]{
    \setlength{\parindent}{4ex}
    \titleformat{\subsection}[display]
    {} % before-code
    {} % header
    {0pt} % vertical or horizontal space
    {} % title
    \titlespacing*{\subsection}{0pt}{-11pt}{-13pt} % remove space for header at all
    {
        \subsection{}{
            \thesubsection. {#1}
        }

    }
}

\newcommand{\mysubsectiom}[1]{
    %\setcounter{secnumdepth}{1-}
    \setlength{\parindent}{4ex}
    \titleformat{\subsection}[display]
    {} % before-code
    {} % header
    {0pt} % vertical or horizontal space
    {} % title
    \titlespacing*{\subsection}{0pt}{-11pt}{-13pt} % remove space for header at all
    {
        \subsection{}{
            {#1}
        }    
    }
}

\setcounter{tocdepth}{1}

\begin{document}

\tableofcontents


\mypart {
ПОЛОЖЕНИЕ
}{
ПОЛОЖЕНИЕ\\ О Совете депутатов Самарcкого внутригородского района\\ городского 
округа Самара
}


\mysection {
Общие положения
}

\mysubsection {
Совет депутатов Самарского внутригородского района городского округа Самара (далее - Совет) является
постоянно действующим выборным представительным органом местного самоуправления Самарского
внутригородского района городского округа Самара (далее - Самарский район).
}

\mysubsection {
Совет состоит из 26 (двадцати шести) депутатов, избираемых на основе всеобщего равного и прямого
избирательного права при тайном голосовании по смешанной избирательной системе с закрытыми списками
кандидатов, при которой половина депутатов избирается по мажоритарной избирательной  системе
относительного большинства по одномандатным избирательным округам, другая половина - по единому
избирательному округу, включающему в себя всю территорию Самарского района, пропорционально числу
голосов избирателей, поданных за списки кандидатов, выдвинутые избирательными объединениями.

Депутаты Совета осуществляют свои полномочия на непостоянной основе.
}

\mysubsection {
Совет является коллегиальным органом, решения которого принимаются голосованием.
}

\mysubsection {
Гарантии осуществления полномочий депутата Совета, Председателя Совета устанавливаются федеральными
законами и Законом Самарской области от 10 июля 2008 года № 67-ГД ``О гарантиях осуществления
полномочий депутата, члена выборного органа местного самоуправления, выборного должностного лица
местного самоуправления в Самарской области'', Уставом Самарского района и муниципальными правовыми
актами Самарского района. 

Депутату Совета для осуществления депутатских полномочий, а также для встреч с избирателями на
территории его избирательного округа Главой Администрации Самарского района безвозмездно
предоставляется отдельное муниципальное помещение.
}

\mysubsection {
Расходы на обеспечение деятельности Совета предусматриваются в бюджете Самарского района отдельной
строкой в соответствии с классификацией расходов бюджетов Российской Федерации. Организационное,
правовое, документальное, информационное, материально-техническое обслуживание деятельности Совета,
Председателя Совета осуществляется структурным подразделением Администрации Самарского
внутригородского района. 
}

\mysection {
Срок полномочий Совета
}

\mysubsection {
Срок полномочий Совета составляет 5 (пять) лет.
}

\mysubsection {
Порядок и основания досрочного прекращения полномочий Совета, депутатов Совета устанавливаются
Федеральным законом от 06 октября 2003 года № 131-ФЗ «Об общих принципах организации местного
самоуправления в Российской Федерации» и Уставом Самарского района.
}

\mysubsection { 
Досрочное прекращение полномочий Совета влечет досрочное прекращение полномочий депутатов Совета и
Председателя Совета.
}

\mysubsection {
В день первого заседания Совета нового созыва полномочия Совета прежнего созыва прекращаются.
}

\mysection {
Порядок работы Совета
}

\mysubsection {
Совет самостоятельно определяет свою структуру, формы организации своей деятельности и правила
организационно-технического обеспечения своей работы в соответствии с принимаемыми Советом
муниципальными правовыми актами.
}

\mysubsection { 
Свою работу Совет осуществляет во взаимодействии с иными органами местного самоуправления Самарского
района, органами местного самоуправления иных муниципальных образований, органами государственной
власти Российской Федерации и Самарской области, органами территориального общественного
самоуправления, общественными объединениями и профессиональными союзами и строит свои отношения с
ними на основе взаимного уважения и конструктивного сотрудничества. 
}

\mysubsection {
Деятельность Совета носит открытый (публичный) характер, который обеспечивается своевременным
доведением до сведения жителей Самарского района соответствующей информации и созданием для них
необходимых условий доступа к депутатам Совета, а также возможности личного присутствия на
заседаниях Совета, комитетов, комиссий и рабочих групп и на иных мероприятиях, организуемых и
проводимых Советом.
}

\mysubsection {
Основной формой работы Совета являются заседания, проводимые на плановой основе. Заседания Совета
проводятся, как правило, не реже одного раза в 2 (два) месяца и по мере необходимости. 

Заседания Совета являются открытыми. По мотивированному протокольному решению, принятому на
соответствующем заседании Совета, заседание либо его часть может быть проведено в закрытом режиме.
Совет может проводить выездные заседания.

Заседание Совета правомочно, если на нем присутствует не менее не менее 14 (четырнадцати) депутатов
Совета.

Депутат Совета обязан принимать личное участие в заседании Совета.

На заседании Совета в обязательном порядке ведется протокол, который подписывается Председателем
Совета и секретарем заседания Совета, избранным на соответствующем заседании.

По вопросам, рассматриваемым на заседании Совета, Президиума Совета, комитетов, комиссий, а также на
иных мероприятиях, проводимых Советом, присутствуют руководители отраслевых (функциональных) органов
Администрации Самарского внутригородского района, в компетенцию которых входит рассмотрение
вопроса.
}

\mysubsection {
Первое заседание Совета нового созыва проводится не позднее, чем на 30 (тридцатый) день после
избрания Совета в правомочном составе.

Первое заседание Совета нового созыва назначается действующим составом Совета в течение 5 (пяти)
дней со дня определения результатов выборов депутатов Самарского района.

На первом заседании Совета нового созыва заслушивается информация о результатах выборов депутатов
Совета, избирается Председатель Совета, формируются комитеты Совета, утверждаются председатели
комитетов Совета.

На первом заседании Совета нового созыва до избрания Председателя Совета председательствует
старейший по возрасту депутат Совета.
}

\mysubsection {
На первом заседании Совета избираются 2 (два) депутата Совета для осуществления полномочий депутатов
Думы городского округа Самара на срок полномочий Думы городского округа Самара.

Решение об избрании депутатов Совета в состав Думы городского округа Самара принимается большинством
голосов от установленной численности депутатов Совета (не менее 14 (четырнадцати) депутатов).

Решение об избрании депутатов Совета в состав Думы городского округа Самара направляется в Думу
городского округа Самара в день его принятия.

В случае, если избранный в состав Думы городского округа Самара, депутат Совета не справляется со
своими обязанностями в Думе городского округа Самара и не посещает официальные мероприятия,
проводимые в Думе городского округа Самара, и иными способами нарушает работу Думы городского округа
Самара, Совет принимает решение об исключении его из состава Думы городского округа Самара по
письменному обращению Думы городского округа Самара.

Рассмотрение вопроса об исключении депутата Совета из состава Думы городского округа Самара
осуществляется на заседании Совета не позднее чем через 30 (тридцать) дней со дня поступления
письменного обращения Думы городского округа Самара.

Решение Совета об исключении депутата Совета из состава Думы городского округа Самара принимается
большинством голосов от установленной численности депутатов Совета (не менее 14 (четырнадцати)
депутатов).

На данном заседании Совета принимается также решение об избрании иного депутата Совета в состав Думы
городского округа Самара взамен выбывшего. 


В случае окончания срока полномочий соответствующего созыва Думы городского округа Самара, на срок
полномочий которого избраны депутаты Совета, Совет принимает решение об избрании депутатов Совета в
состав Думы городского округа Самара нового созыва не позднее чем за 30 (тридцать) дней до дня
окончания срока полномочий соответствующего созыва Думы городского округа Самара.

В случае досрочного прекращения полномочий Думы городского округа Самара соответствующего созыва, в
состав которого избраны депутаты Совета, Совет принимает решение об избрании депутатов Совета в
состав Думы городского округа Самара нового созыва в течение 1 (одного) месяца со дня такого
досрочного прекращения полномочий Думы городского округа Самара.
}

\mysubsection {
Организационное обеспечение первого и последующих заседаний Совета осуществляет уполномоченное
структурное подразделение Администрации Самарского района.
}

\mysection {
Компетенция Совета
}

\mysubsection {
Исключительная компетенция Совета устанавливается Федеральным законом от 06 октября 2003 года №
131-ФЗ «Об общих принципах организации местного самоуправления в Российской Федерации» и Уставом
Самарского района.

Совет также осуществляет иные полномочия, отнесенные к компетенции представительных органов местного
самоуправления федеральными законами, законами Самарской области, Уставом Самарского района.

По вопросам своей компетенции, в соответствии с федеральным законодательством и законодательством
Самарской области, Уставом Самарского района, Совет принимает решения, устанавливающие правила,
обязательные для исполнения на территории Самарского района, а также решения, носящие ненормативный
характер, в том числе об удалении Председателя Совета в отставку, решения по вопросам организации
деятельности Совета.
}

\mysubsection {
Совет заслушивает ежегодные отчеты Главы Администрации Самарского района о его деятельности и
результатах деятельности Администрации Самарского района по форме, утверждаемой решением Совета. 
}

\mysubsection {
Совет является субъектом правотворческой инициативы в Думе городского округа Самара, а также
обладает правом законодательной инициативы в Самарской Губернской Думе.
}

\mysection {
Председатель Совета
}

\mysubsection {
Организацию деятельности Совета осуществляет Председатель Совета.
}

\mysubsection {
Председатель Совета является высшим выборным должностным лицом Самарского района (главой Самарского
района), наделенным Уставом Самарского района в соответствии с Федеральным законом от 06 октября
2003 года № 131-ФЗ «Об общих принципах организации местного самоуправления в Российской Федерации»
собственными полномочиями по решению вопросов местного значения Самарского района, избираемым из
состава Совета на срок полномочий Совета и осуществляющим полномочия Председателя Совета.
}

\mysubsection {
Председатель Совета, организуя деятельность Совета:
\begin{enumerate}[sentence] 
\item представляет Совет в отношениях с органами государственной власти Российской Федерации и
Самарской области, органами местного самоуправления Самарского района, других муниципальных
образований, гражданами и организациями, без доверенности действует от имени Совета
\item созывает заседания Совета путем организации оповещения депутатов о времени и месте проведения
заседаний Совета, направления проекта повестки дня заседания Совета, а также прилагаемых к ней
документов
\item председательствует на заседаниях Совета
\item подписывает решения Совета, протоколы заседаний Совета, иные документы, подготовленные по
результатам проводимых им в Совете мероприятий
\item координирует работу комитетов, комиссий и иных органов Совета
\item организует обеспечение депутатов Совета информацией, необходимой им для осуществления своей
деятельности
\item обеспечивает гласность и учет общественного мнения в работе Совета
\item подписывает от имени Совета исковые заявления, отзывы на исковые заявления, иные
процессуальные документы, а также доверенности на представление интересов Совета
\item осуществляет иные полномочия по организации деятельности Совета в соответствии с действующим
законодательством, настоящим Положением
\end{enumerate}
}

\mysubsection {
Председатель Совета подконтролен и подотчетен населению Самарского района и Совету. 

Председатель Совета представляет не позднее 1 апреля года, следующего за отчетным, Совету ежегодный
отчет о результатах своей деятельности по форме, утверждаемой Советом.

Отчет Председателя Совета и решение Совета о его рассмотрении подлежат официальному опубликованию.
}

\mysubsection {
Председатель Совета избирается на первом заседании Совета нового созыва на срок полномочий Совета.

Избранным Председателем Совета считается депутат Совета, за которого проголосовало не менее 14
(четырнадцати) депутатов Совета.
}

\mysubsection {
Правом выдвижения кандидатур для избрания на должность Председателя Совета обладают депутаты Совета.
При этом каждый депутат Совета вправе выдвинуть только одну кандидатуру на должность Председателя
Совета. Допускается самовыдвижение.
}

\mysubsection {
Избрание Председателя Совета оформляется решением Совета об избрании Председателя Совета, которое
подписывается вновь избранным Председателем Совета и вступает в силу со дня его принятия.
}

\mysubsection {
Порядок избрания Председателя Совета, предусмотренный настоящей статьей, применяется также при
избрании нового Председателя Совета из состава Совета действующего созыва в случае досрочного
прекращения полномочий ранее избранного Председателя Совета.
}

\mysubsection {
Председатель Совета считается вступившим в должность со дня принятия решения Совета о его избрании.

Полномочия Председателя Совета начинаются со дня его вступления в должность и прекращаются в день
вступления в должность вновь избранного Председателя Совета, за исключением случаев досрочного
прекращения полномочий Председателя Совета.
}

\mysubsection {
Полномочия Председателя Совета прекращаются досрочно в случаях, предусмотренных Федеральным законом
от 06 октября 2003 года № 131-ФЗ «Об общих принципах организации местного самоуправления в
Российской Федерации», Уставом Самарского района.
}

\mysection {
Участие депутатов Совета в работе Президиума Совета, комитетов, комиссий, рабочих групп Совета,
депутатских фракций Совета
}

\mysubsection {
Для предварительного рассмотрения вопросов, входящих в компетенцию Совета, подготовки по ним решений
Совета и контроля за их исполнением, а также в целях осуществления контрольных функций, выработки
отдельных позиций, устранения разногласий, решения конкретных задач, рассмотрения (разрешения)
отдельных возникающих в ходе работы Совета вопросов, для подсчетов результатов голосований, для
достижения соглашений, а также по другим вопросам в Совете создаются и формируются комитеты,
комиссии, рабочие группы.
}

\mysubsection {
Для предварительного оперативного рассмотрения вопросов, входящих в компетенцию Совета, координации
деятельности ее комитетов, комиссий и рабочих групп, согласования отдельных проектов решений Совета
и включения их в повестку очередного заседания Совета, разработки предложений по вопросам, выносимым
на заседания Совета, формируется Президиум Совета.
}

\mysubsection {
Депутаты Совета имеют право объединяться в добровольные постоянные фракции при Совете.
}

\mysubsection {
Организационное обеспечение заседаний комитетов, комиссий, рабочих групп, Президиума Совета, фракций
Совета осуществляет уполномоченное структурное подразделение Администрации Самарского района.
}

\mysection {
Комитеты, комиссии, рабочие группы Совета
}

\mysubsection {
Комитеты формируются из числа депутатов Совета на срок полномочий Совета.

Комиссии и рабочие группы формируются на срок рассмотрения вопросов из числа депутатов, сотрудников
Совета. В состав комиссий и рабочих групп могут также входить по согласованию представители иных
органов местного самоуправления Самарского района, органов местного самоуправления иных
муниципальных образований, органов государственной власти Российской Федерации и Самарской области,
юридических лиц, в том числе общественных организаций.
}

\mysubsection {
В Совете по основным направлениям деятельности формируется 5 (пять) комитетов Совета:
\begin{enumerate}[sentence]
\item комитет по бюджету, налогам и экономике
\item комитет по жилищным, имущественным и земельным вопросам
\item комитет по социальным вопросам
\item комитет по местному самоуправлению
\item контрольный комитет
\end{enumerate}
}

\mysubsection {
Комитеты формируются решением Совета, их количественный и персональный состав устанавливается
решением Совета.

Направления деятельности комитетов Совета определяются Приложением к настоящему Положению.
}

\mysubsection {
Каждый депутат Совета может являться членом не более чем 2 (двух) комитетов.

В состав комитета не может входить менее 9 (девяти) и более 11 (одиннадцати) депутатов Совета. В
случае если на вхождение в состав комитета претендует более 11 (одиннадцати) депутатов Совета,
формирование комитета проводится на конкурсной основе либо мягким рейтинговым голосованием.
}

\mysubsection {
Председатель комитета избирается на первом заседании комитета и утверждается на заседании Совета.
}

\mysubsection {
Комиссии и рабочие группы формируются решением Совета (по вопросам, связанным с реализацией органами
местного самоуправления Самарского района вопросов местного значения) или распоряжением Председателя
Совета (по вопросам организации деятельности Совета).
}

\mysection {
Президиум Совета
}

\mysubsection {
К основным направлениям деятельности Президиума Совета относятся:
\begin{enumerate}[sentence]
\item утверждение плана работы Совета
\item назначение даты, места и времени проведения очередного заседания Совета
\item рассмотрение и утверждение проекта повестки дня заседания Совета
\item рассмотрение отдельных проектов решений Совета, предлагаемых для вынесения на очередное
заседание Совета, решение вопроса о включении рассматриваемого проекта решения Совета в повестку
очередного заседания Совета
\item координация деятельности комитетов
\item иные вопросы организации деятельности Совета
\end{enumerate}
}

\mysubsection {
Президиум Совета состоит из Председателя Совета и председателей комитетов Совета.
}

\mysubsection {
Председателем Президиума Совета является Председатель Совета, который руководит работой Президиума
Совета и председательствует на его заседаниях.
}

\mysubsection {
Заседание Президиума Совета правомочно в присутствии не менее половины от установленного состава
Президиума Совета.
}

\mysubsection {
Решения Президиума Совета принимаются большинством голосов от состава Президиума Совета и вступают в
силу со дня их принятия.
}

\mysubsection {
Депутаты, не входящие в состав Президиума Совета, принимают участие в заседаниях Президиума Совета с
правом совещательного голоса.
}

\mysubsection {
Заседания Президиума Совета назначаются распоряжением Председателя Совета.
}

\mysection {
Депутатские фракции
}

\mysubsection {
Состав фракции утверждается решением Совета.
}

\mysubsection {
Порядок работы фракции, принятия фракцией решений, участия фракции в заседаниях Совета определяется
распоряжением Председателя Совета.
}

\mysection {
Помощники депутатов Совета
}

\mysubsection {
Для содействия в осуществлении депутатской деятельности депутат Совета вправе иметь помощников,
осуществляющих свои обязанности без оплаты (на общественных началах).
}

\mysubsection {
Депутат Совета самостоятельно подбирает необходимых ему кандидатов в помощники в количестве не более
3 (трёх) человек, определяет срок их работы.
}

\mysubsection {
На основании представления депутата Совета в соответствии с распоряжением Председателя Совета
помощник депутата Совета ставится на учет в Совете.
}

\mysubsection {
Помощнику депутата Совета выдается удостоверение единого образца, форма которого утверждается
Советом.

Выдача удостоверения помощнику депутата Совета осуществляется лично под роспись в журнале
регистрации выдачи служебных удостоверений в Совете.
}

\mysubsection {
Деятельность помощника депутата Совета может быть прекращена в любой момент по письменному
уведомлению депутата о прекращении деятельности помощника депутата на имя Председателя Совета и
оформляется соответствующим распоряжением Председателя Совета.
}

\mysubsection {
В случае прекращения деятельности помощника депутата Совета его удостоверение подлежит возврату.
}

\mypart {
ОСНОВНЫЕ НАПРАВЛЕНИЯ ДЕЯТЕЛЬНОСТИ КОМИТЕТОВ
}{
ОСНОВНЫЕ НАПРАВЛЕНИЯ ДЕЯТЕЛЬНОСТИ КОМИТЕТОВ
}

\mysectiom {
Комитет по бюджету, налогам и экономике
}

\mysubsectiom {
К основным направлениям деятельности комитета по бюджету, налогам и экономике относится рассмотрение
вопросов:
\begin{itemize}[sentenced]
\renewcommand\labelitemi{-}
\item бюджетного регулирования, в том числе, составления и рассмотрения проекта бюджета Самарского
внутригородского района, утверждения и исполнения бюджета Самарского внутригородского района,
осуществления контроля за его исполнением, составления и утверждения отчета об исполнении бюджета
Самарского внутригородского района:
\item по установлению, изменению и отмене местных налогов и сборов на территории Самарского
внутригородского района в пределах прав, предоставленных законодательством Российской Федерациио
налогах и сборах;
\item содействия деятельности некоммерческих организаций, выражающих интересы субъектов малого и
среднего предпринимательства, и структурных подразделений указанных организаций.
\end{itemize}
}

\mysectiom {
Комитет по жилищным, имущественным и земельным вопросам
}

\mysubsectiom {
К основным направлениям деятельности комитета по жилищным, имущественным и земельным вопросам
относится рассмотрение вопросов:
\begin{itemize}[sentenced]
\renewcommand\labelitemi{-}
\item в сфере жилищно-коммунального хозяйства в соответствии с действующим законодательством
\item в области развития автомобильных дорог местного значения, по содержанию и ремонту
автомобильных дорог местного значения
\item в сфере жилищных правоотношений в соответствии с действующим законодательством, в том числе
осуществления муниципального жилищного контроля
\item владения, пользования и распоряжения имуществом, находящимся в муниципальной собственности
\item обращения с твердыми коммунальными отходами
\item взаимодействия с органами местного самоуправления городского округа Самара при решении ими
вопросов предоставления земельных участков, расположенных в границах внутригородского района, для
целей строительства
\item осуществления муниципального земельного контроля в границах внутригородского района
\item организации благоустройства территории внутригородского района в соответствии с действующим
законодательством
\end{itemize}
}

\mysectiom {
Комитет по социальным вопросам
}

\mysubsectiom {
К основным направлениям деятельности комитета по социальным вопросам относится рассмотрение
вопросов:
\begin{itemize}[sentenced]
\renewcommand\labelitemi{-}
\item создания условий для оказания медицинской помощи населению на территории Самарского
внутригородского района в соответствии с действующим законодательством
\item создания условий для развития местного традиционного народного художественного творчества,
участия в сохранении, возрождении и развитии народных художественных промыслов в Самарском
внутригородском районе
\item создания, развития и обеспечения охраны лечебно-оздоровительных местностей и курортов местного
значения на территории Самарского внутригородского района, а также осуществления муниципального
контроля в области использования и охраны особо охраняемых природных территорий местного значения
\item обеспечения учета детей, подлежащих обучению по образовательным программам дошкольного,
начального общего, основного общего и среднего общего образования, проживающих на территории
Самарского внутригородского района, в соответствии с нормативными правовыми актами городского округа
Самара
\item оказания содействия органам местного самоуправления городского округа Самара в осуществлении
мер по сохранению, использованию и популяризации объектов культурного наследия (памятников истории и
культуры), находящихся в собственности городского округа, охране объектов культурного наследия
(памятников истории и культуры) местного (муниципального) значения, расположенных на территории
Самарского внутригородского района
\item создания условий для организации досуга жителей Самарского внутригородского района
\item обеспечения условий для развития на территории Самарского внутригородского района физической
культуры и массового спорта
\item создания условий для массового отдыха жителей Самарского внутригородского района и организация
обустройства мест массового отдыха населения
\item организации и осуществления мероприятий по работе с детьми и молодежью
\end{itemize}
}

\mysectiom {
Комитет по местному самоуправлению
}

\mysubsectiom {
К основным направлениям деятельности комитета по местному самоуправлению относится рассмотрение
вопросов:
\begin{itemize}[sentenced]
\renewcommand\labelitemi{-}
\item организации местного самоуправления в Самарском внутригородском районе и развития
муниципальной службы
\item осуществления органами местного самоуправления Самарского внутригородского района полномочий
по решению вопросов местного значения Самарского внутригородского района
\item осуществления органами местного самоуправления Самарского внутригородского района отдельных
государственных полномочий
\item разработки и принятия Устава Самарского внутригородского района и внесения в него изменений и
дополнений
\item развития территориального общественного самоуправления и других форм прямой демократии
\end{itemize}
}

\mysectiom {
Контрольный комитет
}

\mysubsectiom {
К основным направлениям деятельности контрольного комитета относится рассмотрение вопросов:
\begin{itemize}[sentenced]
\renewcommand\labelitemi{-}
\item осуществления контроля за исполнением решений Совета депутатов Самарского внутригородского
района
\item соблюдения прав гражданина и человека
\item осуществления муниципального лесного контроля
\item мобилизационной подготовки, пожарной безопасности, осуществления мероприятий в чрезвычайных
ситуациях в соответствии с действующим законодательством
\item создания условий для деятельности добровольных формирований населения по охране общественного
порядка
\end{itemize}
}

\end{document}
}

\textbf{}